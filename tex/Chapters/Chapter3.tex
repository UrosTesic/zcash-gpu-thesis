% Chapter Template

\chapter{zCash and zk-SNARKs} % Main chapter title

\label{Chapter3} % Change X to a consecutive number; for referencing this chapter elsewhere, use \ref{ChapterX}

%----------------------------------------------------------------------------------------
%	SECTION 1
%----------------------------------------------------------------------------------------

\section{zCash}

Cryptocurrencies have gained in popularity over the last couple of years. The best example is the Bitcoin bubble which made people gain millions over night, and lose them soon after. However, Bitcoin is of limited use today. It cannot be used as a general currency due to the low throughput and long waiting times for the transaction to be added to the block. Its status as an anonymous currency is also disputed, resulting in 2 different cryptocurrencies being developed to fill that void - Monero and zCash (ZEC).\\
\\
ZCash defines two types of addresses. Transparent addresses behave the same way as Bitcoin addresses. Data resides in the public blockchain, so it can be tracked the same way as with Bitcoin. However, shielded addresses reveal nothing when they are a part of the transaction. This is accomplished by using notes (NOTE) which contain the public key (PK) of the owner, some amount of zCash (M), as well as a unique identifier (N). Every shielded transaction results in a note like this (transaction output). When this note is used in a transaction, it is spent, and its nullifier (NULL(N)) is published. Valid, unspent notes are ones which are in the set of all generated notes, and whose nullifier hasn't been published.\\
\\
Keeping all notes in a list would result is abysmal performance, so they are kept in a Merkle tree. Furthermore, instead of keeping notes in a public structure, where everyone can see them, we keep commitments to notes (COMM(NOTE)). This guarantees that the value and owner of note are not public. However, now that we don't know PK of the note's owner, how do we verify the transaction?\\
\\
Spending a note in zCash involves computing a zero-knowledge proof $\pi$. It requires the NOTE's spender to prove the following:

\begin{itemize}
    \item The commitment COMM(NOTE) exists in a Merkle tree with all notes
    \item That they have the private key SK, corresponding to the note's public key PK
    \item The NULL(N) is equal to the nullifier provided (T)
\end{itemize}

If the proof verification passes, the node can check if the note has been spent previously by searching for T in a Merkle tree with published nullifiers. This results in the transaction being accepted and added to the blockchain. Along with this T is added to the set of spent nullifiers.

%-----------------------------------
%	SUBSECTION 1
%-----------------------------------
\subsection{Subsection 1}

Nunc posuere quam at lectus tristique eu ultrices augue venenatis. Vestibulum ante ipsum primis in faucibus orci luctus et ultrices posuere cubilia Curae; Aliquam erat volutpat. Vivamus sodales tortor eget quam adipiscing in vulputate ante ullamcorper. Sed eros ante, lacinia et sollicitudin et, aliquam sit amet augue. In hac habitasse platea dictumst.

%-----------------------------------
%	SUBSECTION 2
%-----------------------------------

\subsection{Subsection 2}
Morbi rutrum odio eget arcu adipiscing sodales. Aenean et purus a est pulvinar pellentesque. Cras in elit neque, quis varius elit. Phasellus fringilla, nibh eu tempus venenatis, dolor elit posuere quam, quis adipiscing urna leo nec orci. Sed nec nulla auctor odio aliquet consequat. Ut nec nulla in ante ullamcorper aliquam at sed dolor. Phasellus fermentum magna in augue gravida cursus. Cras sed pretium lorem. Pellentesque eget ornare odio. Proin accumsan, massa viverra cursus pharetra, ipsum nisi lobortis velit, a malesuada dolor lorem eu neque.

%----------------------------------------------------------------------------------------
%	SECTION 2
%----------------------------------------------------------------------------------------

\section{Main Section 2}

Sed ullamcorper quam eu nisl interdum at interdum enim egestas. Aliquam placerat justo sed lectus lobortis ut porta nisl porttitor. Vestibulum mi dolor, lacinia molestie gravida at, tempus vitae ligula. Donec eget quam sapien, in viverra eros. Donec pellentesque justo a massa fringilla non vestibulum metus vestibulum. Vestibulum in orci quis felis tempor lacinia. Vivamus ornare ultrices facilisis. Ut hendrerit volutpat vulputate. Morbi condimentum venenatis augue, id porta ipsum vulputate in. Curabitur luctus tempus justo. Vestibulum risus lectus, adipiscing nec condimentum quis, condimentum nec nisl. Aliquam dictum sagittis velit sed iaculis. Morbi tristique augue sit amet nulla pulvinar id facilisis ligula mollis. Nam elit libero, tincidunt ut aliquam at, molestie in quam. Aenean rhoncus vehicula hendrerit.